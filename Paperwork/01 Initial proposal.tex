\documentclass[11pt]{amsart}
\usepackage{fullpage}
\usepackage{geometry}
\geometry{letterpaper}
\usepackage{graphicx}
\usepackage{amssymb}
\usepackage{epstopdf}
\usepackage{mathpazo} % Use Palatino font instead of Computer Modern
\linespread{1.11} % Palatino needs more leading (space between lines)

\title{Honors proposal}
\author{Noah Morris}
\date{May 25, 2011}

\begin{document}
\maketitle
%\section{}
%\subsection{}

The Aharonov--Bohm effect is perhaps the strangest phenomenon in the strange pantheon of quantum mechanics.\ Following in the footsteps of D.S. Rokhsar's work considering the particle-in-a-box problem \cite{rokhsar}, this project will investigate the AB effect using force ideas, in contrast to the conventional treatment emphasizing potentials.\ In particular, we must ask about the force that the central pillar (dividing the two slits) exerts on the electron, considering the pillar as the limit of a very deep inverted square well.\ (It is clear that the magnetic force on the electron vanishes, but not the pillar force.)\ This theoretical investigation has already been touched upon \cite{peshkin}, but it remains an open question since experimental results on the matter \cite{caprez} that come to contradictory conclusions.

The techniques used will be a combination of computer simulation and old-fashion pencil-and-paper work.\ Supposing, as we hypothesize, that we confirm the results of \cite{peshkin}, we also hope to locate the flaw in the analysis of \cite{caprez} that led them to contradictory conclusions.\\

\begin{thebibliography}{9}

\bibitem{rokhsar}
  D.S. Rokhsar,
  ``Ehrenfest's theorem and the particle-in-a-box,''
  Am. J. Physics 64, 1416.
  November 1996.

\bibitem{peshkin}
  M. Peshkin,
  ``Appendix D: Ehrenfest's theorem,''
  \emph{Lecture Notes in Physics:\ The Aharonov--Bohm Effect}.\ Springer--Verlag,
  1989.

\bibitem{caprez}
  Adam Caprez, Brett Barwick, and Herman Batelaan,
  ``Macroscopic test of the Aharonov--Bohm effect,''
  PRL 99, 210401.\ 23 November 2007.

\end{thebibliography}

\end{document}